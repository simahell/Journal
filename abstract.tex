\documentclass[journal]{IEEEtran}
\input{settings}
\begin{document}
本研究ではウェブの安全性解析に形式手法を用いる。
形式手法ではシステムの仕様等を命題論理で表現したセキュリティモデルが必要となるが、既存のウェブセキュリティモデルではキャッシュが包括されていないことに注目する。
キャッシュは一般的に使用されるウェブの要素であり、これを利用した攻撃法も存在することから安全性解析に不可欠である。
したがって、本研究はキャッシュを包括したウェブセキュリティモデルを実装することを目的とする。

しかし、キャッシュの表現には時相論理が必要である。
そもそも、既存モデルの実装に用いられているAlloy Analyzerで用いる言語Alloyは時相論理を表現する記法を持たない。
このような背景から既存モデルでは独自の記述法で時相論理を導入している。
しかし、これらの記述法で利用できる時相論理の能力は、キャッシュの状態変化の表現には不十分である。
したがって、ウェブの要素の状態変化を表現するために十分な時相論理のAlloy上で利用できる記述法が必要となる。

本研究では、この時相論理の記述法を考案し、それを用いてキャッシュを導入したウェブセキュリティモデルを提案する。
また、キャッシュの基本的な動作やBrowser Cache Poisoning Attackのような既知の攻撃例を事例として、提案モデルの表現能力の向上を確認する。
\end{document}
