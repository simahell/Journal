\documentclass[journal]{IEEEtran}
\input{settings}
\begin{document}
複雑化しているWebの安全性解析に対し、システム仕様を数学的に記述する形式検証を用いることで、汎用的に解析するアプローチが注目されている。
このアプローチは広く利用されている一方、Webにとって必須といえるキャッシュが既存研究には含まれていない。
キャッシュを利用した攻撃法も存在することから、本研究はキャッシュを包括したウェブセキュリティモデルの提案を目的とする。

一般に、キャッシュの表現には時間の流れを表現した時相論理が必要である。
しかしながら、既存モデルの拡張を考えた場合、これは自明ではない。
具体的には、既存モデルの実装に用いられているAlloy Analyzerで用いる言語Alloyは時相論理を表現する記法を持たない。
これに対し既存モデルでは独自の記述法で時間の概念を記述しているが、これらの記述法で利用できる時相論理の能力は、通信が並行して発生する状況を反映できておらず不十分である。
したがって、ウェブの要素の状態変化を表現するために十分な時相論理の記述法をAlloy上でまずは考える必要がある。

本研究では、この時相論理の記述法を考案し、それを用いてキャッシュを導入したウェブセキュリティモデルを提案する。
また、キャッシュの基本的な動作やBrowser Cache Poisoning Attackのような既知の攻撃例を事例として、提案モデルの表現能力の向上を確認する。
\end{document}
