\documentclass[journal]{IEEEtran}
\input{settings}
\begin{document}

\section{準備}

\subsection{形式手法}
\subsubsection{概要}
システムの安全性検査において,数学的な証明を利用する形式手法が存在する.
この手法では,入力となるセキュリティモデルを作成し,それを用いたモデル検査によりシステムの安全性を確認できる.

実際にシステム開発に形式手法を利用する際には以下の手順を行う.
\begin{enumerate}
\item モデル作成\\
設計したシステムをセキュリティモデルとして表現する
\item モデル検査\\
用意したセキュリティモデルに対してモデル検査を実施する
\item 結果分析\\
得られた出力結果からシステムの仕様定義の漏れや脆弱性を発見し,その対策法を考案する
\item モデル修正\\
3.で考案した対策法をセキュリティモデルに反映し,「2.モデル検査」から繰り返す
\end{enumerate}

\subsubsection{セキュリティモデル}
セキュリティモデルは形式手法における入力にあたり,検査対象のシステムを命題論理を用いて表現したものである\cite{security_modeling_and_analysis}.
セキュリティモデルに記述する項目は主に以下の3つである.
\begin{itemize}
\item 対象のシステムの構造と動作
\item 脅威モデル(システムにとって脅威となりうるもののモデル 例:攻撃者の能力)
\item 安全性要件(システムが安全である場合に満たしているべき条件)
\end{itemize}

\subsubsection{Alloy Analyzer}
Alloy Analyzerは形式手法に利用できるモデル検査ツールの1つである.
検査対象のシステムのを表現したセキュリティモデルをAlloyという専用の言語で記述し,これを入力として実行する.
この実行により,モデルが取りうる状態のうち任意のある条件満たしたもの,もしくは,その条件を満たさないものの2通りの出力を得ることができる.
前者は主にセキュリティモデルが正しく実装されていることを確認するための事例研究に利用する.
システムが包括していると想定される事例を条件に記述することで,その事例を含むシステムの挙動が出力され,その事例が正しく表現されていることを確認できる.
一方で,後者の出力結果は主にシステムの安全性解析に用いる.
システムが満たしているべき安全性要件を条件に記述することで,その要件を満たさないシステムの挙動が出力される.
この挙動には安全性要件を侵す脆弱性が含まれているため,この結果の解析によって脆弱性の発見が可能である.

また,Alloy Analyzerが他の形式手法ツールとは異なる点として,実行結果をグラフとして得られるため,より直観的な利用が可能であることが挙げられる.

\subsubsection{時相論理}
\label{sec:TemporalLogic}
時相論理は通常の命題論理に「常に~である」「次の状態では~である」といった,いくつかの論理演算子を加えることで時間変化を表現できるよう拡張したものである.
時相論理において記述した論理式は,その性質から以下の2種類に分類できる.
\begin{itemize}
\item 状態論理式\\
状態論理式はモデルが取りうる各状態に対して真偽が定まる
\item パス論理式\\
パスはそのモデルにおいて状態が変化していく遷移を表す.
パス論理式はモデルが取りうるパスに対して真偽が定まる.
\end{itemize}

また,時相論理にもいくつか種類が存在し,それぞれで扱える演算子が異なるため表現能力が異なる.
以下に形式手法に用いられる3つの時相論理を挙げる.
\begin{itemize}
\item LTL(linear temporal logic / 線形時相論理)\\
LTLはパス論理式のみを表現可能である.
\item CTL(compuational tree logic / 計算木論理)\\
CTLは状態論理式のみを表現可能である.
\item CTL*\\
CTL*はパス論理式,及び,状態論理式のどちらも表現可能である.
また,上記のLTLとCTLはCTL*の部分論理である.
\end{itemize}

\subsection{World Wide Web}
World Wide Web(以下,「ウェブ」とする)はハイパーテキストを記述するHTMLが用いられているドキュメントをインターネット上で提供するシステムである.
ハイパーテキストとはあるドキュメント内に別のドキュメントへのリンクを埋め込み,ドキュメント間の関連性を表現可能とする技術である.
ウェブにおいてはそのリンクを用いてインターネット上に存在する様々なドキュメントの繋がりを表現し,参照先のドキュメントの表示やダウンロードを容易にする.
また,この利便性に加えて,ウェブを利用するためのソフトウェアであるブラウザや検索エンジンの技術向上により,ウェブは急速に普及し現代社会において不可欠なものとなっている.

\end{document}
