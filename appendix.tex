\documentclass[journal]{IEEEtran}
\input{settings}
\begin{document}

\section{Same-origin BCP攻撃を含む結果を出力する検証コード}

\begin{lstlisting}[caption=Same-origin BCP攻撃の表現, label=code:Same_origin_BCP]
run Same_origin_BCP{
	#HTTPClient = 1
	#HTTPServer = 1
	#HTTPIntermediary = 1
	#PrivateCache = 1
	#PublicCache = 0

	#HTTPRequest = 3
	#HTTPResponse = 2
	#CacheReuse = 1

	#Principal = 3
	#Alice = 2

	some tr,tr',tr'':HTTPTransaction | {
		tr'.request.current in tr.request.current.*next
		tr.response.current in tr'.response.current.*next
		tr''.request.current in tr.response.current.*next
		some tr''.re_res

		tr.request.from in HTTPClient
		tr.request.to in HTTPIntermediary

		tr'.request.from in HTTPIntermediary
		tr'.request.to in HTTPServer

		tr''.request.from in HTTPClient

		tr.response.body != tr'.response.body
	}

	some c:HTTPClient | c in Alice.httpClients
	some s:HTTPServer | s in Alice.servers
	no i:HTTPIntermediary | i in Alice.servers
} for 6
\end{lstlisting}

\section{CSRFを含む結果を出力する検証コード}

\begin{lstlisting}[caption=CSRF攻撃の表現, label=code:CSRF]
run CSRF{
	#HTTPRequest = 2
	#HTTPResponse = 2

	#HTTPClient = 1
	#HTTPServer = 2
	#HTTPIntermediary = 0

	#Principal = 3
	#Alice = 2

	all p:Principal |
		one c:HTTPConformist |
			c in p.(servers + httpClients)
	all b:Browser | b in Alice.httpClients

	one tr1,tr2:HTTPTransaction|{
		tr2.request.current in tr1.response.current.*next

		tr1.request.to !in Alice.servers
		tr2.request.to in Alice.servers

		tr2.cause = tr1

		tr1.request.uri != tr2.request.uri
	}
} for 4
\end{lstlisting}

\section{Cross-origin BCP攻撃を含む結果を出力する検証コード}

\begin{lstlisting}[caption=Cross-origin Browser Cache Poisoning攻撃の表現, label=code:Cross_origin_BCP]
run Cross_origin_BCP{
	#HTTPRequest = 5
	#HTTPResponse = 4
	#CacheReuse = 1

	#Browser = 1
	#HTTPServer = 2
	#HTTPProxy = 1
	#Cache = 1

	#Principal = 4
	#Alice = 3

	one Uri

	all c:Cache | c in Browser.cache

	all p:Principal |
		one c:HTTPConformist |
			c in p.(servers + httpClients)
	all b:Browser | b in Alice.httpClients
	all s:HTTPServer | s in Alice.servers

	all tr:HTTPTransaction |{
		tr.request.to in HTTPIntermediary implies{
			one tr':HTTPTransaction |{
				tr'.request.from in HTTPIntermediary
				tr'.request.to in HTTPServer

				tr'.request.current = tr.request.current.next
				tr'.response.current = tr'.request.current.next
				tr.response.current = tr'.response.current.next
			}
		}
	}
	
	one disj tr1,tr3:HTTPTransaction |{
		tr1.request.from in HTTPClient
		tr1.request.to in HTTPIntermediary
		tr3.request.from in HTTPClient
		tr3.request.to in HTTPIntermediary

		tr3.request.current in tr1.response.current.*next
		tr3.cause = tr1
	}

	some tr2,tr4:HTTPTransaction |{
		tr2.request.from in HTTPProxy
		tr2.request.to in HTTPServer
		tr4.request.from in HTTPProxy
		tr4.request.to in HTTPServer
		tr2.request.to != tr4.request.to
	}

	one tr5:HTTPTransaction |{
		tr5.request.from in HTTPClient
		tr5.request.to in HTTPServer
		one tr5.re_res

		all tr:HTTPTransaction | (one tr.response implies tr5.request.current in tr.response.current.*next)
	}

	all tr,tr':HTTPTransaction |{
		{
			tr.request.from in HTTPClient
			tr.request.to in HTTPIntermediary
			tr'.request.from in HTTPIntermediary
			tr'.request.to in HTTPServer
		}implies{
			tr.response.body != tr'.response.body
		}
	}
} for 10
\end{lstlisting}

\section{WCD攻撃を含む結果を出力する検証コード}

\begin{lstlisting}[caption=WCD攻撃の表現, label=code:WCD]
run Web_Cache_Deception{
	#HTTPRequest = 3
	#HTTPResponse = 2
	#CacheReuse = 1

	#HTTPClient = 2
	#HTTPServer = 1
	#HTTPProxy = 1
	#Cache = 1

	#Principal = 4
	#Alice = 3

	all c:Cache | c in HTTPProxy.cache

	all p:Principal |
		one c:HTTPConformist |
			c in p.(servers + httpClients)
	all i:HTTPProxy | i in Alice.servers
	all s:HTTPServer | s in Alice.servers

	one tr1,tr2,tr3:HTTPTransaction |{
		tr1.request.from in Alice.httpClients
		tr1.request.to in HTTPProxy

		tr2.request.from in HTTPProxy
		tr2.request.to in HTTPServer

		(tr3.request.from !in Alice.httpClients and tr3.request.from in HTTPClient)
		tr3.request.to in HTTPProxy

		one tr3.re_res
	}
} for 6
\end{lstlisting}

%\begin{figure}[htb]
%\centering
%\includegraphics[width=450pt]{./fig/WCD_alloy.eps}
%\caption{WCD攻撃のフローを含む状態の一例}
%\label{fig:WCD_alloy}
%\end{figure}

\end{document}
